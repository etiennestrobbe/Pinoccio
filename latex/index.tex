\-Cet exercice produit un simulateur pour un processeur hypothétique et extrêmement simplifié. \-Le processeur lui-\/même est décrit dans la structure \hyperlink{struct_machine}{\-Machine} et ses formats d'instructions dans la structure \hyperlink{union_instruction}{\-Instruction}.\hypertarget{index_code}{}\section{\-Organisation des fichiers source et objets}\label{index_code}
\-Le code est fourni sous forme de fichiers d'entête ({\ttfamily }.h) et d'une bibliothèque binaire ({\bfseries libsimul.\-a}). \-Le seul fichier {\ttfamily }.c fourni ({\ttfamily \hyperlink{test__simul_8c}{test\-\_\-simul.\-c}}) contient la fonction \hyperlink{test__simul_8c_a0ddf1224851353fc92bfbff6f499fa97}{main()}.

\-Le code original complet est composé de 5 modules. \-Chaque module comporte un fichier {\ttfamily }.h et un {\ttfamily }.c de même nom. \-Le fichier-\/source {\ttfamily }.c n'est pas fourni mais le fichier-\/objet correspondant figure dans la bibliothèque {\ttfamily libsimul.\-a}.


\begin{DoxyDescription}
\item[\-Module {\ttfamily machine} (\hyperlink{machine_8h}{machine.\-h}, machine.\-c, machine.\-o) ]\-Ce module décrit la structure générale de la machine préchargée avec un programme et des données. \-Ce module décrit et permet d'initialiser les mémoires d'instruction et de données et d'imprimer l'état courant de la machine (instruction, données, registres). 


\item[\-Module {\ttfamily instruction} (\hyperlink{instruction_8h}{instruction.\-h}, \hyperlink{instruction_8c}{instruction.\-c}, instruction.\-o) ]\-La structure (le format) des instructions de la machine est décrit dans ce module qui fournit aussi une fonction de \char`\"{}désassemblage\char`\"{} (\hyperlink{instruction_8c_a437080d5e8c504588a47d0dba468492f}{print\-\_\-instruction()}) c'est-\/à-\/dire d'impression d'une instruction sous une forme humainement sympathique. 


\item[\-Module {\ttfamily exec} (\hyperlink{exec_8h}{exec.\-h}, \hyperlink{exec_8c}{exec.\-c}, exec.\-o) ]\-On trouve dans ce module le code permettant le décodage et l'exécution des instructions. 


\item[\-Module {\ttfamily error} (\hyperlink{error_8h}{error.\-h}, \hyperlink{error_8c}{error.\-c}, error.\-o) ]\-C'est le module d'affichage (en clair) des messages d'erreurs et autre {\itshape warnings\/}. 


\item[\-Module {\ttfamily debug} (\hyperlink{debug_8h}{debug.\-h}, \hyperlink{debug_8c}{debug.\-c}, debug.\-o) ]\-Ce module permet l'exécution interactive en pas à pas. \-Sa fonction \hyperlink{debug_8c_a3a88fdc680b7a1ae8c4c7c8ddee730ab}{debug\-\_\-ask()} est invoquée après l'exécution de chaque instruction de la machine et gère un dialogue permettant à l'utilisateur d'afficher l'état de la machine (contenu des mémoires et des registres) ou de passer à l'exécution de l'instruction suivante. 


\item[\-Fichier {\ttfamily \hyperlink{test__simul_8c}{test\-\_\-simul.\-c}}  ]\-Ce fichier source contient la fonction \hyperlink{test__simul_8c_a0ddf1224851353fc92bfbff6f499fa97}{main()} qui


\begin{DoxyItemize}
\item décode les options de ligne de commande (voir plus loin)


\item initialise la machine avec le programme et les données initiales


\item affiche l'état (mémoires, registres) initial de la machine


\item exécute complètement le programme


\item affiche l'état (mémoires, registres) final de la machine 
\end{DoxyItemize}

\-Les options de la ligne de commande sont 
\begin{DoxyDescription}
\item[-\/h ]\-Affiche un message d'aide (\char`\"{}help\char`\"{}).


\item[-\/d ]\-Lance l'exécution en mode interactif pas à pas (\char`\"{}debug\char`\"{}).


\item[-\/b ]\-Le dernier argument de la ligne de commande doit être le nom d'un fichier {\itshape binaire\/} contenant une représentation du programme et de ses données. \-Le format de ce fichier est décrit avec la fonction \hyperlink{machine_8h_ac59b88844961c2479108151e24dd555a}{read\-\_\-program()}. \-On en trouvera des exemples dans le repertoire \-Examples (fichiers {\ttfamily }.bin).

\-Sans option {\bfseries -\/b} la fonction \hyperlink{test__simul_8c_a0ddf1224851353fc92bfbff6f499fa97}{main()} choisit et exécute un programme prédéfini (dans le fichier {\ttfamily prog.\-o} de la bibliothèque {\ttfamily libsimul.\-a}).






\end{DoxyDescription}

\begin{DoxyAttention}{\-Attention}
{\itshape \-Le code est écrit en langage \-C et utilise la norme \-C99 (option {\bfseries -\/std=c99} de {\bfseries gcc}). \-Il ne compile pas en mode \-C90 !\/}
\end{DoxyAttention}

\end{DoxyDescription}\hypertarget{index_make}{}\section{\-Utilisation de la Makefile}\label{index_make}

\begin{DoxyDescription}
\item[make ]\-Reconstruit l'exécutable de test, {\bfseries test\-\_\-simul}. 


\item[make doc ]\-Reconstruit la documentation html dans doc/html. \-Requiert \href{http://www.doxygen.org}{\tt {\bfseries doxygen}. }


\item[make clean  ]\-Détruit tous les fichiers objets générés mais pas l'exécutable de test ni la documentation.


\item[make clobber  ]\-Détruit tous les fichiers générés y compris l'exécutable de test (mais pas la documentation).


\item[make clean\-\_\-doc ]\-Détruit la documentation générée par {\bfseries doxygen}.  
\end{DoxyDescription}\hypertarget{index_build}{}\section{\-Construction et utilisation du simulateur}\label{index_build}
\hypertarget{index_build_init}{}\subsection{\-Construction du simulateur initial}\label{index_build_init}
\-Initialement il suffit de faire sous un {\bfseries shell} quelconque 
\begin{DoxyCode}
make
\end{DoxyCode}
 puis 
\begin{DoxyCode}
test_simul
\end{DoxyCode}
 pour exécuter le programme prédéfini. \-Ou alors 
\begin{DoxyCode}
test_simul -d -b Examples/prog_subroutine.bin
\end{DoxyCode}
 pour exécuter le programme dont le nom est indiqué en mode interactif.\hypertarget{index_build_yours}{}\subsection{\-Intégration de votre propre code}\label{index_build_yours}
\-Pour intégrer votre propre code, il suffit d'éditer la \-Makefile pour ajouter vos propres fichiers {\ttfamily }.c ; pour cela il vous suffit de définir la variable de {\bfseries make} nommée {\ttfamily \-U\-S\-E\-R\-S\-R\-C} en y mettant la liste de vos propres fichiers {\ttfamily }.c (elle est initialement vide) ; par exemple


\begin{DoxyCode}
USERSRC = machine.c debug.c 
\end{DoxyCode}


puis de faire {\bfseries make} pour reconstruire l'exécutable {\bfseries test\-\_\-simul}. \-Cela compilera vos {\ttfamily }.c et l'éditeur de liens choisira les {\ttfamily }.o correspondants de préférence à ceux qui sont dans la bibliothèque {\ttfamily libsimul.\-a}. \-Si vous ajoutez des {\ttfamily }.h ils seront pris en compte par la regénération des dépendances ({\ttfamily depend.\-out}).

\begin{DoxyAttention}{\-Attention}
{\itshape {\bfseries \-Ne modifiez pas les fichiers {\ttfamily }.h fournis car les objets de {\ttfamily libsimul.\-a} en dépendent}\/}.
\end{DoxyAttention}
\begin{DoxyAuthor}{\-Author}
\-Jean-\/\-Paul \-Rigault 
\end{DoxyAuthor}
\begin{DoxyDate}{\-Date}
\-Avril 2011 
\end{DoxyDate}
